\chapter{Appendix}
\label{chap:appendix}
\section{Review Questions \& Responses}
\begin{enumerate}[label=Q\arabic*.]
    % Ghada's Questions
    \item \textit{Unclear context of OT-related cybersecurity incidents (p. 2)} --- The context of the cybersecurity incidents was clarified by rephrasing the sentence and moving the citation to a more appropriate position.
    \item \textit{Textual citation of 3 or more authors should be shortened to et al. (p. 2 ff.)} --- The maximum number of author names in textual citations was limited to two, e.g., (Elbez, Keller, and Hagenmeyer [16]) becomes (Elbez et al. [16]).
    \item \textit{Typo in the word therefore (p. 2)} --- A missing letter was added to the word therefore.
    \item \textit{Research questions should relate to previously described challenges, and RQ1 and RQ2 should differ more from each other (p. 2)} --- All three research questions were partially reconstructed to relate to the security objectives. Moreover, the research questions were rephrased to clarify their goals.
    \item \textit{Interchanged words in contribution section (p. 3)} --- The corresponding sentence was rephrased.
    \item \textit{To which extent is the SABAAC approach flexible (p. 3)} --- Since its policies and not the approach is expressive and flexible, the sentence was rephrased. Moreover, the expressiveness, flexibility, and computationally expensiveness of the policies were further clarified.
    \item \textit{It would be good to relate the fundamentals to the field of application (p. 6)} --- Since the fundamentals were shortened from twenty to two pages, the relations between its topics and the OT/ICS/SAS-domain were lost. Moreover, the relevance of the more abstract topics including access control and cryptography for the OT-domain is not sufficiently described in the proposal. This will be added to the fundamentals chapter of the thesis, as the realization is currently an open question.
    \item \textit{A paragraph could be added to relate the related work chapter to the fundamentals (p. 7)} --- This relation is present in the long thesis proposal in the form of a division into sections, which are related to topics of the fundamentals.
    \item \textit{Please consider organizing the related work into sections (p. 7)} --- This organization is present in the long thesis proposal and will be transferred to the final master's thesis. Moreover, an extension for the cryptography-specific related work is planned.
    \item \textit{Please consider discussing some limitations of related works on ABAC and RBAC to justify the motivation of your work (p. 8)} --- The similarities and differences of the thesis and related work as well as the limitations of related work are present in the long thesis proposal.
    \item \textit{Cybersecurity and cryptography are names of CASC-SAS layers (p. 9)} --- The differences of the two layers are discussed in detail in the long thesis proposal. However, since the two words cybersecurity and cryptography are closely related, renaming or renumbering of the layers is currently under consideration.
    \item \textit{For ease of read, better present requirements as a list (p. 10)} --- In the long thesis proposal the requirements and their short descriptions are presented as a list.
    \item \textit{How does the integration of CASA effectively relate to figure 4.2 (p. 13)} --- This figure caption was shortened, and the CASA-specific part was removed for better understandability.
    \item \textit{Which adversarial attacks would be relevant as use cases (p. 16)} --- A subsection in the system model dedicated to relevant attacks and attack trees is planned for the final thesis. Moreover, relevant attacks were added to the short proposal's requirements section. Due to the limited space in the short proposal, no attack-specific section with enumerated attacks was added.
    \item \textit{Which considerations would be included in the economic evaluation (p. 16)} --- Besides the aspects mentioned in the proposal, metrics such as the cost of CASC-SAS equipment will be discussed in the economic evaluation. Moreover, the economic evaluation covers the compatibility-related requirements, including interoperability and interchangeability, which are core concepts of the IEC 61850.
    \item \textit{Which evaluation aspects are possible theoretically (p. 16)} --- The theoretical evaluation covers the security proofs of CASA, the economic evaluation, and the calculation of minimum transfer time requirements for the performance evaluation. The theoretical approaches used will be discussed in detail in the corresponding evaluation sections in the thesis.
    \item \textit{Are times in the work plan total times and are increments processed in parallel (p. 17)} --- To clarify the total durations of milestones and parallel execution of increments, a sentence was added to the work plan introduction.
    \item \textit{Shouldn't software design flaws be considered before software implementation flaws (p. 20)} --- The order of the risks was changed to clarify that the design should be flawless before the implementation.
    % Ramadan's Questions
    \item \textit{The figure captions are too long} --- All figure captions in the proposal were shortened.
    \item \textit{The fundamentals are lacking a cryptography section} --- A cryptography section covering symmetric and asymmetric cryptography was added to the fundamentals of the long proposal.
    \item \textit{The NIST recommendations for access control are not discussed in the fundamentals} --- A NIST recommendations section was added to the fundamentals of the long proposal.
    \item \textit{The security model of CASA makes very strong statements about EUF-CMA, which are not correct anymore} --- The strong statement was removed from the security model and the section was rephrased.
    \item \textit{The introduction chapter lacks an objective section} --- A section for the objective of the thesis was added to the introduction chapter. Moreover, the already existing research question section was integrated into the new section.
    \item \textit{The introduction chapter lacks a thesis structure section} --- A section for the proposed structure of the thesis was added to the introduction chapter of the long proposal. Moreover, a graphical representation of the structure is currently under consideration.
    \item \textit{The typical cybersecurity incidents mentioned in related work are not present in the objective of the approach} --- A paragraph for cybersecurity incidents was added to the objective section.
    % Meeting Questions
    \item \textit{Sender and receiver are interchanged in SABAAC figures} --- The errors in the SABAAC figures were fixed.
    \item \textit{SABAAC is misspelled in the time schedule} --- The error in the time schedule figure was fixed.
    \item \textit{A deprecated security requirement is present in the CASA description in the long proposal} --- The deprecated security requirement was removed.
    \item \textit{Prof. Dr. Veit Hagenmeyer is the first examiner of the thesis} --- The examiners on the title page of the short and long proposal were changed. However, the second examiner is currently unknown.
    \item \textit{Missing group logo at title page} --- The KASTEL logo was added to the title page as group logo. Note: This change was reverted for the template version of the proposal.
    \item \textit{Citations should be numbered in the order in which they appear} --- The ordering of the bibliography was changed.
    \item \textit{State-sponsored cybersecurity incidents included summary of incidents from 2013 to 2020 (p. 2)} --- The mentioning of non-discussed but referenced incidents from 2013 to 2020 was removed.
    \item \textit{Ordering of research questions is not consistent with main thesis focus ABAC (p. 2)} --- Changed order of RQ1 and RQ2.
    \item \textit{RQ2 and RQ3 contain concepts of contribution and shrink the solution space too much (p. 2)} --- Revised research questions by replacing concrete concepts such as certificateless and server-aided with more goal-oriented and unbiased terms such as lightweight and scalable.
    \item \textit{Introduction chapter lacks an enumeration of thesis contributions (p. 3)} --- The contribution section was rephrased and restructured. The contributions are now presented as an enumeration.
    \item \textit{The asymmetric cryptography section should be renamed to PKC and lacks a description of symmetric cryptography (p. 5)} --- The PKC section was renamed and a sentence to distinguish symmetric from asymmetric algorithms was added. The long version of the proposal contains an own section for symmetric cryptography.
    \item \textit{The economic evaluation is not possible due to missing data and the term economic/economy should be avoided (p. 16)} --- The third evaluation area was renamed to compatibility, to emphasize the new focus on interoperability and interchangeability as defined in IEC 61850, and the focus on economy in this area was reduced by rephrasing and restructuring.
    \item \textit{Overlapping increments of time schedule do not seem feasible (p. 19)} --- The realization phase was extended, and the evaluation phase was shortened by two weeks. Moreover, the increment durations and starting dates were revised to reduce overlapping phases.
    \item \textit{The title page should present meta information differently} --- The title page text and the thesis TeX class were changed.
    \item \textit{Project plan chapter should be renamed to thesis plan} --- The chapter was renamed. Moreover, the term project was replaced with the term thesis at appropriate locations in the text.
\end{enumerate}
